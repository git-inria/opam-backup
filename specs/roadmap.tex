\documentclass[a4paper,11pt]{article}

\usepackage{url}
\usepackage{eurosym}
%\usepackage[french]{babel}
\usepackage[T1]{fontenc}
\usepackage{pdfswitch}
\usepackage{verbatim}
\usepackage{fullpage}

\newenvironment{remark}[1][Remark]{\begin{trivlist}
\item[\hskip \labelsep {\bfseries #1}]}{\end{trivlist}}

\title{
OPAM: a Package Management Systems for OCaml\\
Version 1.0.0 Roadmap\\ ~\ \\
THIS DOCUMENT IS A DRAFT\\
~\ \\}
\author{Thomas GAZAGNAIRE\\
\url{thomas.gazagnaire@ocamlpro.com}\\
}

\begin{document}

\maketitle

\vfill

\tableofcontents

\section*{Overview}

This document specifies the design of a package management system for
OCaml (OPAM). For the first version of OPAM, we have tried to consider
the simplest design choices, even if these choices restrict user
possibilities (but we hope not too much). Our goal is to propose a
system that we can build in a few months. Some of the design choices
might evolve to more complex tasks later, if needed. \\

A package management system has typically two kinds of users: {\em
  end-users} who install and use packages for their own projects; and
{\em packagers}, who create and upload packages. End-users want to
install on their machine a consistent collection of {\em packages} --
a package being a collection of OCaml libraries and/or programs.
Packagers want to take a collection of their own libraries and
programs and make them available to other developpers.

This document describes the fonctional requirements for both kinds of
users.

\subsection*{Conventions}

In this document, {\tt \$home}, {\tt \$opam}, {\tt \$opamserver} and
{\tt \$package} are assumed to be defined as follows:

\begin{itemize}

\item {\tt \$home} refers to the end-user home path, typically {\tt
  /home/thomas/} on linux, {\tt /Users/thomas/} on OSX {\tt
  C:\textbackslash Documents and Settings\textbackslash
  thomas\textbackslash} on Windows.

\item {\tt \$opam} refers to the filesystem subtree containing the
  client state. Default directory is {\tt \$home/.opam}.

\item {\tt \$opamserver} refers to the filesystem subtree containing
  the server state. Default directory is {\tt \$home/.opam-server}.

\item {\tt \$package} refers to a path in the packager filesystem, where
  lives the collection of libraries and programs he wants to package.

\end{itemize}

User variables are written in capital letters, prefixed by \verb+$+. For
instance package names will be written \verb+$NAME+, package versions
\verb+$VERSION+, and the version of the ocaml compiler currently
installed \verb+$OVERSION+.

\section{Milestone 1: Foundations}

The first milestone of OPAM focuses on providing a limited set of
features, dedicated to package management of OCaml packages. OPAM rely
on external tools to compile and provide full configuration options to
the build tools. The goal for this first milestone is to be compatible
with {\tt ocamlfind} and {\tt oasis}.

\subsection{Client state}
\label{client}

The client state is stored on the filesystem, under {\tt \$opam}:

\begin{itemize}

\item {\tt \$opam/config} is the main configuration file. It defines
  the OPAM version, the repository addresses and the current compiler
  version. The file format is described in \S\ref{config}.

\item \verb+$opam/index/$NAME.$VERSION.opam+ is the OPAM specification
  for the package \verb+$NAME+ with version \verb+$VERSION+ (which
  might not be installed). The format of OPAM files is described in
  \S\ref{opam}.

\item \verb+$opam/descr/$NAME.$VERSION+ is the textual description of
  the version \verb+$VERSION+ of package \verb+$NAME+ (which might not
  be installed). The first line of this file is the package synopsis.

\item \verb+$opam/$OVERSION/installed+ is the list of installed
  packages for the compiler version \verb+$OVERSION+. The file format
  is described in \S\ref{installed}.

\item \verb+$opam/$OVERSION/config/$NAME.config+ is a
  platform-specific configuration file of for the installed package
  \verb+$NAME+ with the compiler version \verb+$OVERSION+. The file
  format is described in \S\ref{pconfig}.
  \verb+$opam/$OVERSION/config/+ can be shortened to \verb+$config/+
  for more readability.

\item \verb+$opam/$OVERSION/install/$NAME.install+ is a
  platform-specific package installation file for the installed
  package \verb+$NAME+ with the compiler version \verb+$OVERSION+. The
  file format is described in \S\ref{install}.
  \verb+$opam/$OVERSION/install+ can be shortened to \verb+$install/+
  for more readability.

\item \verb+$opam/$OVERSION/lib/$NAME/+ contains the libraries
  associated to the installed package \verb+$NAME+ with the compiler
  version \verb+$OVERSION+. \verb+$opam/$OVERSION/lib/+ can be
  shortened to \verb+$lib/+ for more readability.

\item \verb+$opam/$OVERSION/doc/$NAME/+ contains the documentation
  associated to the installed package {\tt NAME} with the compiler
  version \verb+$OVERSION+. /\verb+$opam/OVERSION/doc/+ can be
  shortened to \verb+$doc/+ for more readability.

\item \verb+$opam/$OVERSION/bin/+ contains the program files for all
  installed packages with the compiler version
  \verb+$OVERSION+. \verb+$opam/$OVERSION/bin/+ can be shortened to
  \verb+$bin/+ for more readability.

\item \verb+$opam/archives/$NAME.$VERSION.tar.gz+ contains the archive
  of source files for the version \verb+$VERSION+ of package
  \verb+$NAME+.

\item \verb+$opam/$OVERSION/build/$NAME.$VERSION/+ is a tempory folder
  used to build package \verb+$NAME+ with version \verb+$VERSION+,
  with compiler version \verb+$OVERSION+.\verb+$opam/$OVERSION/build/+
  can be shortened to \verb+$build/+ for more readability.

\end{itemize}

\subsection{File Syntax}

\subsubsection{Installed packages}
\label{installed}

\verb+$opam/$OVERSION/installed+ follows a very simple syntax: the
file is a list of lines which each has a name and a version, separated
by a single space. Each line \verb+$NAME $VERSION+ means that the
version \verb+$VERSION+ of package \verb+$NAME+ has been compiled with
OCaml version \verb+$OVERSION+ and has been installed on the system in
\verb+$lib/$NAME+ and \verb+$bin/+.

\subsubsection{General Syntax}
\label{syntax}

Most of the files in the client and server states share the same
syntax defined in this section.

\begin{description}

\item[Base types] The base types are:

\begin{itemize}
\item {\tt BOOL} either {\tt true} or {\tt false}
\item {\tt STRING} a doubly-quoted OCaml string, for instance: {\tt "foo"}
\item {\tt SYMBOL} a symbol contains only non-letter and non-digit
  characters, for instance: {\tt <=}
\item {\tt IDENT} an ident starts by a letter and is followed by any
  number of letters, digit and symbols, for instance: {\tt foo-bar}
\end{itemize}

\item[Compound types] Values of base types can be composed together to
  build complex types:

\begin{itemize}
\item {\tt [ X ]} a space-separated list of values of type {\tt X}
\item {\tt ( X )} a space-separated optional list of values of type
  {\tt X}
\item \verb+{ X }+ a space-separated collection of values of types
      {\tt X} (whose order is thus not meaningful).
\end{itemize}

\item[Files] All structured OPAM files share the same syntax:

\begin{itemize}

\item A file is a space-separated list of {\tt item}s

\item An {\tt item} is either:
\begin{itemize}
\item {\tt IDENT = value}
\item \verb+IDENT STRING { item }+
\end{itemize}

\item a {\tt value} is either:
\begin{itemize}
\item {\tt BOOL}
\item {\tt STRING}
\item {\tt SYMBOL}
\item {\tt [ VALUE ]}
\item \verb+VALUE ( VALUE )+
\end{itemize}

\end{itemize}
\end{description}

\subsubsection{Configuration files}
\label{config}

\verb+$opam/config+ has the following format:

\begin{verbatim}
opam-version = "1.0"
sources = [ STRING ]
ocaml-version = STRING
\end{verbatim}

The field {\tt sources} contains the list of OPAM repositories
(default is {\tt "http://opam.ocamlpro.com/pub/"}). Initially, the
field {\tt ocaml-version} corresponds to the output of {\tt 'ocamlc
  -version'}.

\subsubsection{OPAM files}
\label{opam}

\verb+$opam/index/$NAME.$VERSION.opam+ follows the syntax defined in
\S\ref{syntax}, restricted to the following subset:

\begin{verbatim}
opam-version = 1.0

package NAME {
  version     = STRING
  maintainer  = STRING
  build       = [ STRING ]
  depends     = VALUE
  conflicts   = VALUE
  libraries   = [ STRING ]
  syntax      = [ STRING ]
}
\end{verbatim}

\begin{itemize}

\item The first line specifies the OPAM version.

\item The contents of {\tt version} is {\tt VERSION}. {\tt maintainer}
  contains the contact address of the package maintainer.

\item The content of {\tt build} is the command to run in order to
  build the package libraries. The build script should build all the
  libraries and syntax extensions exported by the package, but it
  should also build and run the tests (if any) and build the
  documentation (if any).

\item The {\tt depends} and {\tt conflicts} fields contain expressions
  over package names, optionally parametrized by version
  constrains. An expression is either:

\begin{itemize}
\item A package name: {\tt "foo"};
\item A package name with version constraints:
  \verb+"foo" (>= "1.2" & <= "3.4")+
\item A disjunction of expressions: \verb+E | F+
\item A conjunction of expressions: \verb+E & F+
\item An expression with parenthesis: \verb+( E )+
\end{itemize}

For instance \verb+ "foo" (<= "1.2") & ("bar" | "gna" (= "3.14"))+
is a valid formula whose semantic is: {\em a version of package
  {\tt "foo"} lesser or equal to $1.2$ and either any version of
  package {\tt "bar"} or the version $3.14$ of package {\tt "gna"}.}
\\

\item The {\tt libraries} and {\tt syntax} fields contain the
  libraries and syntax extensions defined by the package.

\end{itemize}

\subsubsection{Configuration files}
\label{pconfig}

\verb+$opam/OVERSION/config/NAME.config+ follows the syntax defined in
\S\ref{syntax}, restricted to the following subset:

\begin{verbatim}
library STRING {
  include  = [ STRING ]
  asmlink  = [ STRING ]
  bytelink = [ STRING ]
  requires = [ STRING ( STRING ) ]
  pp       = [ STRING ( STRING ) ]
  IDENT = BOOL
  IDENT = STRING
  IDENT = [ STRING ]
  ...
}

syntax STRING {
  include  = [ STRING ]
  asmlink  = [ STRING ]
  bytelink = [ STRING ]
  requires = [ STRING ( STRING ) ]
  pp       = [ STRING ( STRING ) ]
  IDENT = BOOL
  IDENT = STRING
  IDENT = [ STRING ]
  ...
}

IDENT = BOOL
IDENT = STRING
IDENT = [ STRING ]
...
\end{verbatim}

Each {\tt library} block defines the compilation options to use when
linking with this library. Options associated to library dependencies
are dynamically computed using the {\tt requires} field.\\

Each {\tt syntax} block defines the pre-processing options to use when
using this syntax extension. Options associated to pre-processor
dependencies are dynamically computed using the {\tt pp} field. The
full pre-processor command line is then passed to the compiler with
{\tt -pp}.\\

More details on the semantic of each field:

\begin{itemize}

\item {\tt include} is the list of directory to open when compiling a
  project using the library (or the syntax extension). It should at
  least contain \verb+[ "-I" "/full/path/to/NAME" ]+.

\item {\tt asmlink} is either the list of libraries to use when
  linking a project in native code with the library. It should at
  least contain \verb+[ "-I" "/full/path/to/NAME" "NAME.cmxa" ]+

\item {\tt asmlink} is either the list of libraries to use when
  linking a project in byte code with the library. It should at least
  contain \verb+[ "-I" "/full/path/to/NAME" "NAME.cma" ]+

\item {\tt requires} is the list of libraries which needs to be linked
  with the current one. The syntax \verb+"foo" ("bar" "gna")+ means
  only libraries {\tt "bar"} and {\tt "gna"} in package "foo" will be
  considered. The syntax {\tt "foo"} means {\em all} libraries in
  package {\tt "foo"} will be considered.

\item {\tt pp} is the list of syntax extension to use when compiling a
  program using the library. The syntax is similar to {\tt
    requires}. Once expended, the list of arguments is used with the
  {\tt -pp} command-line option of the chosen compiler.

\end{itemize}

The remaining fields {\tt IDENT = BOOL}, {\tt IDENT = STRING} or {\tt
  IDENT = [ STRING ]}, where {\tt IDENT} are in {\em capitalized
  letters}, are used to defined global variables associated to this
package, and are used to substitute variables in template files:

\begin{itemize}
\item \verb+%{$NAME}$VAR%+ will refer to the variable \verb+$VAR+
  defined the file \verb+$config/NAME.config+
\item \verb+%{$NAME.$LIB}$VAR%+ will refer to the variable \verb+$VAR+
  defined in the library or syntax section \verb+$LIB+ in the file
  \verb+$config/$NAME.config+.
\end{itemize}

\subsubsection{Install files}
\label{install}

\verb+$opam/OVERSION/install/NAME.install+ follows the syntax defined
in \S\ref{syntax}, but restricted to the following subset:

\begin{verbatim}
lib  = [ STRING ]
bin  = [ STRING ( STRING ) ]
doc  = [ STRING ]
misc = [ [ STRING STRING ] ]
\end{verbatim}

Files listed under {\tt lib} are copied to \verb+$lib/$NAME/+. File
listed under {\tt bin} are copied to \verb+$bin/+.
Files listed under {\tt doc} are copied to \verb+$doc/$NAME/+.
Files listed under {\tt misc} should be processed as
follows: for each pair \verb+$FILE $DST+, the tool should ask the user if
he wants to install \verb+$FILE+ to the absolute path \verb+$DST+.

\subsubsection{Substitution files}
\label{subst}

All of the previous files can be generated using a special mode of
{\tt opam} which can perform tests and substitution variables (see
\S\ref{opam-config}). The syntax defined in \S\ref{syntax} is extended
as follows:

\begin{itemize}
\item A value can also be of the form \verb+%{IDENT}IDENT%+. The
 semantic of \verb+%{PACKAGE}VAR%+ is the value of the variable {\tt
 VAR} defined in \verb+$config/PACKAGE.config+
\item A value can also be of the form
  \verb+%{IDENT}{IDENT}IDENT%+. The semantic of
  \verb+%{PACKAGE}{LIB}VAR%+ is the value of the variable {\tt VAR}
  defined in the {\tt LIB} section in \verb+$config/PACKAGE.config$+
\item An item can also be of the form \verb+IF test { item }+ or
  \verb+IF test { item } THEN { item }+
  where {\tt test} is:
\begin{itemize}
\item either a value of type {\tt BOOL}
\item or a variable whose value is a boolean
\item or an expression {\tt VALUE = VALUE} where the contents of both
  values is compared structurally
\end{itemize}
\end{itemize}

\subsection{Server state}

The filesystem of OPAM repositories are mirrored on the client
filesystem under \verb$opamserver/$HOSTNAME+ for each remote
repository named \verb+$HOSTNAME+. This filesystem contains:

\begin{itemize}

\item \verb+$opamserver/$HOSTNAME/index/$NAME.$VERSION.opam+, which are
  OPAM files for all available versions of all available packages. The
  format of specification files is described in \S\ref{opam}.

\item \verb+$opamserver/$HOSTNAME/descr/$NAME.$VERSION+, which are
  textual description files for all available versions of all
  available packages.

\item \verb+$opamserver/update+ is the script which will be run each
  time the repository is updated. It should return a list of lines
  following the same format as described in \S\ref{installed} and
  containing the new packages.

\item \verb+$opamserver/getArchive+ is a script which will be run to
  get an the archive corresponding to a package. It takes as argument
  a package name and a package version and it returns the downloaded
  filename.

\item \verb+$opamserver/setArchive+ is a script which will be run
  to upload a new archive. It takes as argument a package name, a
  package version and the archive to upload.

\end{itemize}

\subsection{Client commands}

\subsubsection{Creating a fresh client state}

When an end-user starts OPAM for the first time, he needs to
initialize \verb+$opam/+ in a consistent state. In order to do so, he
should run:

\begin{verbatim}
    $ opam init HOSTNAMES
\end{verbatim}

Where {\tt HOSTNAMES} is a (possibly empty) list of OPAM
repositories. If no OPAM repository is specified, default is
\verb+"opam://opam.ocamlpro.com"+.

This command will:

\begin{enumerate}

\item create the file \verb+$opam/config+ (as specified in
  \S\ref{pconfig})

\item create an empty \verb+$opam/$OVERSION/installed+ file.

\item ask all the server in sequence for all available packages using
  {\tt getList} (\S\ref{getList}) and get all the corresponding spec
  files using {\tt getOPAM} (\S\ref{getOPAM}).

\item dump all the spec files into
  \verb+$opam/index/$NAME.$VERSION.opam+.

\item create empty directories \verb+$opam/archives+; and create
  \verb+$lib/+ and \verb+$bin/+ if they do not exist.

\end{enumerate}

\subsubsection{Listing packages}

When an end-user wants to have information on all available packages,
he should run:

\begin{verbatim}
    $ opam list
\end{verbatim}

This command will parse \verb+$opam/OVERSION/installed+ to know the
installed packages, and \verb+$opam/index/*.opam+ to get all the
available packages. It will then build a summary of each packages. For
instance, if {\tt batteries} version {\tt 1.1.3} is installed, {\tt
  ounit} version {\tt 2.3+dev} is installed and {\tt camomille} is not
installed, then running the previous command should display:

\begin{verbatim}
    batteries   1.1.3  Batteries is a standard library replacement
    ounit     2.3+dev  Test framework
    camomille      --  Unicode support
\end{verbatim}


In case the end-user wants a more details view of a specific package,
he should run:

\begin{verbatim}
    $ opam info NAME
\end{verbatim}

This command will parse \verb+$opam/$OVERSION/installed+ to get the
installed version of \verb+$NAME+ and will look for
\verb+$opam/index/$NAME.*.opam+ to get available versions of
\verb+$NAME+. It can then display:

\begin{verbatim}
    package:  $NAME
    version:  $VERSION                 # '--' if not installed
    versions: $VERSION1, $VERSION2, ...
    libraries: $LIB1, $LIB2, ...
    syntax: $SYNTAX1, $SYNTAX2, ...
    description:
      $SYNOPSIS

      $LINE1
      $LINE2
      $LINE3
\end{verbatim}

\subsubsection{Installing a package}
\label{opam-install}

When an end-user wants to install a new package, he should run:

\begin{verbatim}
    $ opam install NAME
\end{verbatim}

This command will:

\begin{enumerate}

\item look into \verb+$opam/index/$NAME.*.opam+ to find the latest
  version of the package.

\item compute the transitive closure of dependencies and conflicts of
  packages using the dependency solver (see \S\ref{deps}). If the
  dependency solver returns more than one answer, the tool will ask
  the user to pick one, otherwise it will proceed directly.

\item the dependency solver should have sorted the collections of
  packages in topological order. Them, for each of them do:

\begin{enumerate}

\item check whether the package archive is installed by looking for
  the line \verb+$NAME $VERSION+ in \verb+$opam/$OVERSION/installed+.
  If not, then:

\begin{enumerate}

\item look into the archive cache to see whether it has already been
  downloaded. The cache location is: {\tt
    \$opam/archives/NAME.VERSION.tar.gz}.

\item if not, then download the archive and store it in the cache.

\item decompress the archive into \verb+$build/+. By convention, we
  assume that this should create \verb+$build/$NAME.$VERSION/+.

\item run \verb+$build/$NAME.$VERSION/build.sh+. By convention, package
  archives should contains such a file.

\item process {\tt
  \$build/NAME.VERSION/NAME.install}\label{NAME.install}.  The file
  format is described in \S\ref{install}.

\end{enumerate}
\end{enumerate}
\end{enumerate}

\begin{remark}
This installation scheme is not always correct, as installing a new
package should uninstall all packages depending on that one. For
instance, let us consider 3 packages {\tt A}, {\tt B} and {\tt C};
{\tt B} and {\tt C} depend on {\tt A}; {\tt C} depends on {\tt B}.
{\tt A} and {\tt B} are installed, and the user request {\tt C} to be
installed. If the version of {\tt A} is not correct one but the
version of {\tt B} is, the tool should: install the latest version of
{\tt A}, recompile {\tt B}, compile {\tt C}. It is understood that,
with this first milestone, {\tt B} will not be recompiled. This issue
will be fixed in next milestones of OPAM.
\end{remark}

\subsubsection{Updating index files}

When an end-user wants to know what are the latest packages available,
he will write:

\begin{verbatim}
    $ opam update
\end{verbatim}

This command will ask the server the list of available packages using
{\tt getList} (see \S\ref{getList}); then ask for the missing OPAM
files using {\tt getOPAM} (see \S\ref{getOPAM}). Finally it will dump
the missing OPAM files into \verb+$opam/index/$NAME.$VERSION.opam+.

\subsubsection{Upgrading installed packages}

When an end-user wants to upgrade the packages installed on his host,
he will write:

\begin{verbatim}
    $ opam upgrade
\end{verbatim}

This command will call the dependency solver (see \S\ref{deps}) to
find a consistent state where {\em most} of the installed packages are
upgraded to their latest version. It will install each non-installed
packages in topological order, similar to what it is done during the
install step, See \S\ref{install}.

\subsubsection{Getting package configuration}
\label{opam-config}

The first version of OPAM contains the minimal information to be able
to use installed libraries. In order to do so, the end-user (or the
packager) should run:

\begin{verbatim}
    $ opam config -list-vars
    $ opam config -var {$NAME}$VAR+
    $ opam config -var {$NAME.$LIB}$VAR+
    $ opam config -subst $FILENAME+
\end{verbatim}

%%   $ opam config [-r] -I $NAME+
%%   $ opam config [-r] -bytelink $NAME.$LIB+
%%   $ opam config [-r] -asmlink $NAME.$LIB+
%%   $ opam config [-r] -pp $NAME.$LIB+
%%   $ opam config -ocp $NAME

\begin{itemize}

\item \verb+-list-vars+ will return the list of all variables defined
  in installed packages (see \S\ref{pconfig})
\item \verb+-var {$NAME}$VAR+ will return the value of the variable
  \verb+$VAR+ defined in \verb+$config/NAME.config+
\item \verb+-var {$NAME.$LIB}$VAR+ will return the value of the
  variable \verb+$VAR+ defined in the section \verb+$LIB+ in
  \verb+$config/NAME.config+
\item \verb+-subst $FILENAME+ replace any occurrence of
  \verb+%{$NAME}$VAR%+ and \verb+%{$NAME.$LIB}$VAR%+ from
  \verb+$FILENAME.in+ to create \verb+$FILENAME+.
%% \item \verb+-I $NAME+ will return the list of paths to include when
%%   compiling a project using the package \verb+$NAME+ (\verb+-r+ gives
%%   a result taking into account the transitive closure of dependencies).
%% \item \verb+-bytelink $NAME.LIB+ and \verb+-asmlink $NAME.$LIB$+ will
%%   return the list of libraries to include with linking with a project
%%   using the library \verb+$LIB+ in package \verb+$NAME+.

%% \item the file {\tt FILENAME} where every occurrence of
%%   \verb+%{NAME:VAR%}+ in {\tt FILENAME.in} is replaced by its content
\end{itemize}


\subsubsection{Uploading packages}
\label{upload}

When a packager wants to create a package, he should:

\begin{enumerate}

\item create \verb+$package/$NAME.$VERSION.opam+ containing in the format
  specified in \S\ref{opam}.

\item create \verb+$package/$NAME.install+ containing the list of files
  to install. File format is described in \ref{NAME.install});
  filnames should be relative to \verb+$package+.

\item create the script {\tt ./build.sh} which will be called by the
  end-user installer. This script should configure and build the
  package on the end-user host.

\item create an archive \verb+$NAME.$VERSION.tar.gz+ of the sources he
  wants to distribute.

\item run the following command:

\begin{verbatim}
    $ opam upload $FILENAME $ARCHIVE
\end{verbatim}

This command looks into the current directory for a file named 
\verb+$FILE+, and it will parse it to get the package name and version
number. Then it checks that \verb+$ARCHIVE$+ corresponds to
\verb+$NAME.$VERSION.tar.gz+. It will then use the server API~\ref{api} to
upload the package on the server.

This command will work only for READ-WRITE repositories.

\end{enumerate}

\subsection{Removing packages}

When the user wants to remove a package, he should write:

\begin{verbatim}
    $ opam-remove $NAME
\end{verbatim}

This command will check whether the package \verb+$NAME+ is installed,
and if yes, it will display to the user the list packages that will be
uninstalled (ie. the transitive closure of all forward-dependencies).
If the user accepts the list, all the packages should be uninstalled,
and the client state should be let in a consistent state.

\subsection{Dependency solver}
\label{deps}

Dependency solving is a hard problem and we do not plan to start from
scratch implementing a new SAT solver. Thus our plan to integrate (as
a library) the Debian depency solver for CUDF files, which is written
in OCaml.

\begin{itemize}
\item the dependency solver should run on the client; and
\item the dependency solver should take as input a list of packages
  (with some optional version information) the user wants to install,
  upgrade and remove and it should return a consistent list of
  packages (with version numbers) to install, upgrade, recompile and
  remove.
\end{itemize}

\subsection{Getting package options}

The user should be able to run:

\begin{verbatim}
    XXX
\end{verbatim}

This command will return the list of link options to pass to {\tt
  ocamlc} when linking with libraries exported by {\tt NAME}.

In order to be able to do so, packagers should provide a file {\tt
  NAME.descr} which gives link information such as:

\begin{verbatim}
    library foo {
      requires: bar, gni
      link: -linkall
      asmlink: -cclib -lfoo
     }
\end{verbatim}

\subsection{Getting package recursive configuration}

The user should be able to run:

\begin{verbatim}
    $ opam-config -r -dir NAME
    $ opam-config -r -bytelink NAME
    $ opam-config -r -asmlink NAME
\end{verbatim}

This command will return the good options to use for package {\tt
  NAME} and all its dependencies, in a form suitable to be used by
OCaml compilers.

\section{Milestone 2: Custom Client-Server Protocol}
\label{api}

All the kinds of OPAM repositories should be available using the same
API (however, some functions will not be available for some backends).

\subsubsection{Getting the list of packages}

\label{getList}
\begin{verbatim}
    val getList   : repo -> (name * version) list 
    val updateList: repo -> (name * version) list
\end{verbatim}

\verb+getList $HOSTNAME+ returns the full list of available packages. This
command is intended to be run only once, when the repository
\verb+$HOSTNAME+ is initialized.\\

\verb+updateList $HOSTNAME+ updates the given repository and returns
the list of newly available packages. For repositories not using the
custom OPAM protocol (eg. not starting by \verb+opam://+) this means
running the script \verb+$opamserver/$HOSTNAME/update+ which should
return a list of lines of (package name, version) pairs using the same
format as described in \S\ref{installed}.

\subsubsection{Getting OPAM files}
\label{getOPAM}

\begin{verbatim}
    val getOPAM: repo -> (name * version) -> opam
\end{verbatim}

{\tt getOPAM repo (name,version)} returns the corresponding OPAM
filename as an absolute location in the filesystem (which should be .

\subsubsection{Getting description files}
\label{getOPAM}

\begin{verbatim}
    val getDescr: repo -> (name * version) -> descr
\end{verbatim}

{\tt getDescr repo (name,version)} returns the corresponding
description file.

\subsubsection{Getting package archive}
\label{getArchive}

\begin{verbatim}
    val getArchive: repo -> (name * version) -> archive
\end{verbatim}

{\tt getArchive repo (name,version)} returns the corresponding package
archive.

\subsubsection{Uploading new archives}
\label{newArchive}

\begin{verbatim}
    val newArchive: repo -> (opam * archive) -> unit
\end{verbatim}

{\tt newArchive(opam,archive)} takes as input an OPAM file and the
corresponding package archive, and upload the server state. This
function works only for READ-WRITE repository. In case of a READ-ONLY
one, a suitable error message is returned to the user.

\subsubsection{Binary Protocol}

In case of READ-WRITE repositories, the server state can be queried
and modified by any OPAM clients, using the following binary protocol

\begin{itemize}

\item Communication between clients and servers always start by an
hand-shake to agree on the protocol version.

\item All the basic values (names, versions and binary data) are
  represented as OCaml strings.

\item More complex values are marshaled using a simple binary
  protocol: the first byte represents the message number, and then
  each message argument is stacked in the message with its size as
  prefix. The list of messages {\em from the client to server} is:

{\small
\begin{tabular}{|l|l|l|}
\hline
Client-to-Server Message & Arguments & Description \\
\hline
\hline
\verb+GetList+ & -- & Ask for the list of all OPAM files \\
\hline
\verb+GetOPAM+ & \verb+name   : string+ & Ask for the binary representation of \\
               & \verb+version: string+ & a given OPAM file \\
\hline
\verb+GetArchive+ & \verb+name   : string+ & Ask for the binary representation of \\
                  & \verb+version: string+ & a given archive file \\
\hline
\verb+NewArchive+ & \verb+name   : string+ & Create a new package on the
server. \\
                  & \verb+version: string+ & The client should provide
the OPAM file \\
                  & \verb+opam   : string+ & and the source archive. \\
                  & \verb+archive: string+ & \\
\hline
\verb+UpdateArchive+ &  \verb+name   : string+ & Update a new version
of a given \\
                  & \verb+version: string+ & package on the
server. The client \\
                  & \verb+opam   : string+ & should also provide a security key\\
                  & \verb+archive: string+ & \\
                  & \verb+key    : string+ & \\
\hline
\end{tabular}
}

\item Answers from the server are encoded in the same way (ie, a byte
  for the message number, followed by optional arguments prefixed by
  their size). List arguments are encoded by stacking first the
  lenght, and then all the elements of the list in sequential order.
  The list of messages {\em from servers to clients} is:

{\small
\begin{tabular}{|l|l|l|}
\hline
Server-to-Client Message & Arguments & Description \\
\hline
\hline
\verb+GetList+      & \verb+list   : (string*string) list+ & Return the list
of available \\
 & & package names and versions \\
\hline
\verb+GetOPAM+      & \verb+opam   : string+ & Return an OPAM file \\
\hline
\verb+GetArchivwe+  & \verb+archive: string+ & Return an archive file \\
\hline
\verb+NewArchive+   & \verb+key    : string+ & Return a security key \\
\hline
\verb+UpdateArchive+& --                     & The update went OK \\
\hline
\verb+Error+        & \verb+error  : string+ & An error occurred \\
\hline
\end{tabular}
}

\end{itemize}

Note that when an error is raised by an arbitrary function
 at server side, the client receives \verb|Error _|.



\section{Milestone 3: Link Information}

This milestone focuses on adding the right level of linking
information, in order to be able to use packages more easily.


\section{Milestone 4: Server Authentication}

This version focuses on server authentication.

\subsection{RPC protocol}

The protocol should be specified (using either a binary format or a
JSON format).

\subsection{Server authentication}

The server should be able to ask for basic credential proofs. The
protocol can be sketched as follows:

\begin{itemize}

\item packagers store keys in {\tt \$opam/keys/NAME}. These keys are
  random strings of size 128.

\item the server stores key hashes in {\tt
  \$opamserver/hashes/NAME}.

\item when a packager wants to upload a fresh package, he still uses
  {\tt newArchive}. However, the return type of this function is
  changed in order to return a random key. OPAM clients then stores
  that key in {\tt \$opam/keys/NAME}.

\item when a packager wants to uplaod a new version of an existing
  package, he uses the function {\tt val updateArchive: (opam * string
    * string) -> bool}. {\tt updateArchive} takes as argument an OCaml
  value representing the OPAM file contents, the archive file as a
  binary string and the key as a string. The server then checks
  whether the hash of the key is equal to the one stored in {\tt
    \$opamserver/hashes/NAME}; if yes, it updates the
  package and return {\tt true}, if no if it returns {\tt false}.

\item packager email should be specified in {\tt NAME.opam}:

\end{itemize}

\section{Milestones 6: Pre-Processors Information}

This milestone focus on the support of pre-processors.

\subsection{Getting package preprocessor options}

The user should be able to run:

\begin{verbatim}
    $ opam-config -bytepp NAME
    $ opam-config -asmpp NAME
\end{verbatim}

This command will return the command line option to build the
preprocessor exported by package {\tt NAME}.

In order to do so, packagers should describe exported preprocessors in
the corresponding {\tt NAME}.descr:

\begin{verbatim}
syntax foo {
  requires: bar, gni         // list of syntax dependencies
  pp: -parser o -printer p   // common options to asmpp and bytepp
  bytepp: ...
}
\end{verbatim}

\section{Milestones 7: Support of Multiple Compiler Versions}

This milestone focus on the support of multiple compiler versions.

\subsection{Compiler Description Files}

For each compiler version {\tt OVERSION}, the client and server states
will be extended with the following files:

\begin{itemize}
\item {\tt \$opam/compilers/OVERSION.comp}
\item {\tt \$opamserver/compilers/OVERSION.comp}
\end{itemize}

Each {\tt .comp} file contains:

\begin{itemize}

\item the location where this version can be downloaded. It can be an
  archive available via {\tt http} or using CVS such as {\tt svn} or
  {\tt git}.

\item eventual options to pass to the configure script. {\tt
  --prefix=\$opam/OVERSION/} will be automatically added to these
  options.

\item options to pass to {\tt make}.

\item eventual patch address, available via {\tt http} or locally on
  the filesystem

\end{itemize}

For instance, {\tt 3.12.1+memprof.comp} (OCaml version $3.12.1$ with
the memory profiling patch) looks like:

\begin{verbatim}
src:       http://caml.inria.fr/pub/distrib/ocaml-3.12/ocaml-3.12.1.tar.gz
build:     world world.opt
patches:   http://bozman.cagdas.free.fr/documents/ocamlmemprof-3.12.0.patch
\end{verbatim}

And {\tt trunk-tk-byte.comp} (OCaml from SVN trunk, with no {\em tk}
support and only in bytecode) looks like:

\begin{verbatim}
src:       http://caml.inria.fr/pub/distrib/ocaml-3.12/ocaml-3.12.1.tar.gz
configure: -no-tk
build:     world
\end{verbatim}

\subsection{Milestone 8: Version Pinning}

\subsection{Milestones 9: Parallel Build}

\subsection{Milestone 10: Version Comparison Scheme}

\subsection{Milestone 11: Database of Installed Files}


\end{document}
