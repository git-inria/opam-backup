\documentclass[a4paper,11pt]{article}

\usepackage{url}
\usepackage{eurosym}
%\usepackage[french]{babel}
\usepackage[T1]{fontenc}
\usepackage{pdfswitch}
\usepackage{verbatim}
\usepackage{fullpage}
\usepackage{fancyvrb}

\newenvironment{remark}[1][Remark]{\begin{trivlist}
\item[\hskip \labelsep {\bfseries #1}]}{\end{trivlist}}

\title{
OPAM: a Package Management Systems for OCaml\\
Version 1.0.0 Roadmap\\ ~\ \\
THIS DOCUMENT IS A DRAFT\\
~\ \\}
\author{Thomas GAZAGNAIRE\\
\url{thomas.gazagnaire@ocamlpro.com}\\
}

\begin{document}

\maketitle

\vfill

\tableofcontents

\section*{Overview}

This document specifies the design of a package management system for
OCaml (OPAM). For the first version of OPAM, we have tried to consider
the simplest design choices, even if these choices restrict user
possibilities (but we hope not too much). Our goal is to propose a
system that we can build in a few months. Some of the design choices
might evolve to more complex tasks later, if needed. \\

A package management system has typically two kinds of users: {\em
  end-users} who install and use packages for their own projects; and
{\em packagers}, who create and upload packages. End-users want to
install on their machine a consistent collection of {\em packages} --
a package being a collection of OCaml libraries and/or programs.
Packagers want to take a collection of their own libraries and
programs and make them available to other developpers.

This document describes the fonctional requirements for both kinds of
users.

\subsection*{Conventions}

In this document, \verb+$home+, \verb+$opam+, \verb+$package+ and 
\verb+$path+ are assumed to be defined as follows:

\begin{itemize}

\item {\tt \$home} refers to the end-user home path, typically {\tt
  /home/thomas/} on linux, {\tt /Users/thomas/} on OSX {\tt
  C:\textbackslash Documents and Settings\textbackslash
  thomas\textbackslash} on Windows.

\item {\tt \$opam} refers to the filesystem subtree containing the
  client state. Default directory is {\tt \$home/.opam}.

\item {\tt \$package} refers to a path in the packager filesystem, where
  lives the collection of libraries and programs he wants to package.

\item {\tt \$path} refers to a list of paths in the packager filesystem, where
  lives the collection of programs ({\tt ocamlc}, {\tt ocamldep}, {\tt ocamlopt}, 
  {\tt ocamlbuild}, ...).

\end{itemize}

User variables are written in capital letters, prefixed by \verb+$+. For
instance package names will be written \verb+$NAME+, package versions
\verb+$VERSION+, and the version of the ocaml compiler currently
installed \verb+$OVERSION+.

\section{Milestone 1: Foundations}

The first milestone of OPAM focuses on providing a limited set of
features, dedicated to package management of OCaml packages. OPAM rely
on external tools to compile and provide full configuration options to
the build tools. The goal for this first milestone is to be as much as
possible compatible with any existing build system (including {\tt
  ocamlfind} and {\tt oasis}) modulo few modifications.

\subsection{Client state}
\label{client}

The client state is stored on the filesystem, under {\tt \$opam}:

\subsubsection{Global state}
\label{state-global}

\begin{itemize}

\item {\tt \$opam/config} is the main configuration file. It defines
  the OPAM version, the repository addresses and the current compiler
  version. The file format is described in \S\ref{config}.

\item \verb+$opam/opam/$NAME.$VERSION.opam+ is the OPAM specification
  for the package \verb+$NAME+ with version \verb+$VERSION+ (which
  might not be installed). The format of OPAM files is described in
  \S\ref{dotopam}.

\item \verb+$opam/descr/$NAME.$VERSION+ contains the description for the
  version \verb+$VERSION+ of package \verb+$NAME+ (which might not be
  installed). The first line of this file is the package synopsis.

\item \verb+$opam/archives/$NAME.$VERSION.tar.gz+ contains the source
  archives for the version \verb+$VERSION+ of package \verb+$NAME+.

\item \verb+$opam/repo/index+ contains the location of packages
  (ie. in which repositories they can be found and the priority
  between repositories). The file format is described in
  \ref{index}.

\end{itemize}

\subsubsection{Compiler-specific state}

All the configurations files, libraries and binaries related to the
specific \verb+$OVERSION+ of the OCaml compiler are stored in
\verb+$opam/$OVERSION+.

\begin{itemize}

\item \verb+$opam/$OVERSION/installed+ is the list of installed
  packages for the compiler version \verb+$OVERSION+. The file format
  is described in \S\ref{installed}.

\item \verb+$opam/$OVERSION/config/$NAME.config+ is a
  platform-specific configuration file of for the installed package
  \verb+$NAME+ with the compiler version \verb+$OVERSION+. The file
  format is described in \S\ref{dotconfig}.
  \verb+$opam/$OVERSION/config/+ can be shortened to \verb+$config/+
  for more readability.

\item \verb+$opam/$OVERSION/install/$NAME.install+ is a
  platform-specific package installation file for the installed
  package \verb+$NAME+ with the compiler version \verb+$OVERSION+. The
  file format is described in \S\ref{dotinstall}.
  \verb+$opam/$OVERSION/install+ can be shortened to \verb+$install/+
  for more readability.

\item \verb+$opam/$OVERSION/lib/$NAME/+ contains the libraries
  associated to the installed package \verb+$NAME+ with the compiler
  version \verb+$OVERSION+. \verb+$opam/$OVERSION/lib/+ can be
  shortened to \verb+$lib/+ for more readability.

\item \verb+$opam/$OVERSION/doc/$NAME/+ contains the documentation
  associated to the installed package {\tt NAME} with the compiler
  version \verb+$OVERSION+. \verb+$opam/OVERSION/doc/+ can be
  shortened to \verb+$doc/+ for more readability.

\item \verb+$opam/$OVERSION/bin/+ contains the program files for all
  installed packages with the compiler version
  \verb+$OVERSION+. \verb+$opam/$OVERSION/bin/+ can be shortened to
  \verb+$bin/+ for more readability.

\item \verb+$opam/$OVERSION/build/$NAME.$VERSION/+ is a tempory folder
  used to build package \verb+$NAME+ with version \verb+$VERSION+,
  with compiler version \verb+$OVERSION+. \verb+$opam/$OVERSION/build/+
  can be shortened to \verb+$build/+ for more readability.

\item \verb+$opam/$OVERSION/reinstall+ contains the list of packages
  which has been changed upstream since the last upgrade. This can
  happen for instance when a packager uploads a new archive or fix the
  OPAM file for a specific package version. Every package appearing in
  this file will be reinstalled (or upgraded if a new version is
  available) during the next upgrade when the current version of the
  compiler is \verb+$OVERSION+. The file format is similar to the one
  described in \S\ref{installed}.

\end{itemize}

\subsubsection{Repository-specific state}
\label{state-repo}

Configuration files for OPAM repositories \verb+REPO+ are stored in
\verb+$opam/repo/$REPO+. Repositories can be of different kinds
(stored on the local filesystem, available via HTTP, available using a
custom binary protocol, stored under git, $\ldots$); they all share
the same base filesystem which is initialized using the {\tt
  opam-<kind>-init} script (see \S\ref{script-init}) and use only the
{\tt opam-<kind>-update} (see \S\ref{script-update}) and {\tt
  opam-<kind>-upload} (see \S\ref{script-upload}) scripts to exchange
data between the client and the corresponding OPAM repository.

\begin{itemize}

\item \verb+$opam/repo/$REPO/kind+ contains the kind associated to the
  OPAM repository \verb+$REPO+ have. The kind is stored as a single
  word (containing only letters and digits) and specifies which {\tt
    opam-<kind>-*} scripts to call when updating and uploading this
  repository.

\item \verb+$opam/repo/$REPO/address+ contains the address of the
  OPAM repository \verb+$REPO+. This address is passed as argument to
  the {\tt opam-<kind>-*} scripts.

\item \verb+$opam/repo/$REPO/opam/$NAME.$VERSION.opam+ is the OPAM
  specification for the package \verb+$NAME+ with version
  \verb+$VERSION+ (which might not be installed). The format of OPAM
  files is described in \S\ref{dotopam}.

\item \verb+$opam/repo/$REPO/descr/$NAME.$VERSION+ contains the textual
  description for the version \verb+$VERSION+ of package \verb+$NAME+
  (which might not be installed). The first line of this file is the
  package synopsis.

\item \verb+$opam/repo/$REPO/archives/$NAME.$VERSION.tar.gz+ contains
  the source archives for the version \verb+$VERSION+ of package
  \verb+$NAME+. This folder is populated by calling the corresponding
  {\tt opam-<kind>-download} script (see \S\ref{script-download}).

\item \verb+$opam/repo/$REPO/updated+ contains the new available
  packages which have not yet been synchronized with the client
  state. This file is created by the {\tt opam-<kind>-update} script
  (see \S\ref{script-update}). If the file empty, this means that the
  client state is up-to-date. The file format is the same as the one
  described in \S\ref{installed}.

\item \verb+$opam/repo/$REPO/upload/$NAME.$VERSION/+ contains the
  OPAM, description and archive files to upload to the OPAM repository
  for the version \verb+$VERSION+ of package \verb+$NAME+. The script
  {\tt opam-<kind>-update} script (see \S\ref{script-upload}) read the
  contents of {\tt upload/} and send it to the repository (if the
  repository support upload).

\end{itemize}

\subsection{File syntax}

\subsubsection{List of packages}
\label{installed}

The following configuration files: \verb+$opam/$OVERSION/installed+,
\verb+$opam/$OVERSION/reinstall+, and \verb+$opam/repo/$REPO/updated+
follow a very simple syntax. The file is a list of lines which
contains a space-separated name and a version. Each line
\verb+$NAME $VERSION+ means that the version \verb+$VERSION+ of
package \verb+$NAME+ has been compiled with OCaml version
\verb+$OVERSION+ and has been installed on the system in
\verb+$lib/$NAME+ and \verb+$bin/+. \\

For instance, if {\tt batteries} version {\tt 1.0+beta} and {\tt
  ocamlfind} version {\tt 1.2} are installed, then
\verb+$opam/$OVERSION/installed+ will contain:

{\small
\begin{Verbatim}[frame=single]
batteries 1.0+beta
ocamlfind 1.2
\end{Verbatim}
}

\subsubsection{Index of packages}
\label{index}

\verb+$opam/repo/index+ follows a very simple syntax: each line of the
file contains a space separated list of words \verb+$NAME $REPO+
specifying that all the versions of package \verb+$NAME+ are available
in the OPAM repository \verb+$REPO+. The file contains information on
all available packages (e.g. not only on the installed one). \\

For instance, if {\tt batteries} version {\tt 1.0+beta} is available
in the {\tt testing} repository and {\tt ocamlfind} version {\tt 1.2}
is available in the {\tt default} and testing repositories (where {\tt
  default} is one being used), then \verb+$opam/repo/index+ will
contain:

{\small
\begin{Verbatim}[frame=single]
batteries testing
ocamlfind default
\end{Verbatim}
}

\subsubsection{General syntax}
\label{syntax}

Most of the files in the client and server states share the same
syntax defined in this section.

\begin{description}

\item[Base types] The base types for values are:

\begin{itemize}
\item {\tt BOOL} is  either {\tt true} or {\tt false}
\item {\tt STRING} is a doubly-quoted OCaml string, for instance: {\tt
  "foo"}, {\tt "foo-bar"}, $\ldots$
\item {\tt SYMBOL} contains only non-letter and non-digit characters,
  for instance: {\tt =}, {\tt <=}, $\ldots$ Some symbols have a special
  meaning and thus are not valid {\tt SYMBOL}s: ``\verb+(+ \verb+)+
  \verb+[+ \verb+]+ \verb+{+ \verb+}+ \verb+:+''.
\item {\tt IDENT} starts by a letter and is followed by any number of
  letters, digit and symbols, for instance: {\tt foo}, {\tt foo-bar},
  $\ldots$. 
\end{itemize}


\item[Compound types] Types can be composed together to build more
  complex values:

\begin{itemize}
\item {\tt X Y } is a space-separated pair of value.
\item {\tt X | Y } is a value of type either {\tt X} or {\tt Y}.
\item {\tt ?X} is zero or one occurrence of a value of type {\tt X}.
\item {\tt X+} is a space-separated list of values of at least one value
  of type {\tt X}.
\item {\tt X*} is a space-separated list of values of values of type
  {\tt X} (it might contain no value).
\end{itemize}

\end{description}

All structured OPAM files share the same syntax:

{\small
\begin{Verbatim}[frame=single]
<file>  := <item>*

<item>  := IDENT : <value>
         | ?IDENT: <value>
         | IDENT STRING { <item>+ }

<value> := BOOL
         | STRING
         | SYMBOL
         | IDENT
         | [ <value>+ ]
         | value { <value>+ }
\end{Verbatim}
}

\subsubsection{Global configuration file}
\label{config}

\verb+$opam/config+ follows the syntax defined in \S\ref{syntax} with
the following restrictions:

{\small
\begin{Verbatim}[frame=single]
<file> :=
    opam-version: "1"
    ?repositories: [ <repo>+ ]
    ocaml-version: STRING

<repo> := STRING { STRING }
\end{Verbatim}
}

The field {\tt repositories} contains the list of OPAM
repositories with their kind.

The field {\tt ocaml-version} corresponds to the current OCaml
compiler (available in the path).

\subsubsection{Package specification files: {\tt .opam}}
\label{dotopam}

\verb+$opam/opam/$NAME.$VERSION.opam+ follows the syntax defined in
\S\ref{syntax} with the following restrictions:

{\small
\begin{Verbatim}[frame=single]
<file> :=
    opam-version: "1"
    package STRING {
      version:    STRING
      maintainer: STRING
      ?subst:     [ STRING+ ]
      ?build:     [ command+ ]
      ?depends:   <or-formula>
      ?conflicts: <and-formula>
      ?libraries: [ STRING+ ]
      ?syntax:    [ STRING+ ]
    }

<or-formula>  := <or-formula> '|' <or-formula>
               | ( <or-formula> )
               | <and-formula>

<and-formula> := STRING
               | STRING { <constraints> }
               | <and-formula> <and-formula>

<constraints> := <comp> STRING
               | <constraints> <constraints>

<comp>       := '=' | '<' | '>' | '>=' | '<='
<command>    := [ STRING+ ]
\end{Verbatim}
}

\begin{itemize}

\item The first line specifies the OPAM version.

\item The string after {\tt package} should not contain any dot ({\tt
  '.'}) nor space ({\tt ' '}).

\item The contents of {\tt version} is \verb+$VERSION+. The content of
  {\tt maintainer} is the contact address of the package maintainer.

\item The content of {\tt subst} is the list of files to substitute
  variable (see \S\ref{subst} for the file format and
  \S\ref{opam-config} for the semantic of file substitution).

\item The content of {\tt build} is the list of commands to run in
  order to build the package libraries. The build script should build
  all the libraries and syntax extensions exported by the package and
  it should produce the platform-specific configuration and install
  files (e.g. \verb+$NAME.config+ and \verb+$NAME.install+, see
  \S\ref{dotconfig} and \S\ref{dotinstall}).

\item The {\tt depends} and {\tt conflicts} fields contain
  respectively {\em or-} and {\em and-}formulas over package names,
  optionally parametrized by version constraints. An expression is
  either:

\begin{itemize}
\item A package name: {\tt "foo"};
\item A package name with version constraints:
  \verb+"foo" (>= "1.2" & <= "3.4")+
\item A disjunction of formulas: \verb+E | F+
\item A conjunction of formulas: \verb+E F+
\item A formula between parenthesis: \verb+( E )+
\end{itemize}

For instance \verb+ "foo" {<= "1.2"} | ("bar" "gna" {= "3.14"})+ is a
valid formula whose semantic is: {\em either any version of package
  {\tt "foo"} lesser or equal to $1.2$ or both any version of package
  {\tt "bar"} and the version $3.14$ of package {\tt "gna"}.}  \\

\item The {\tt libraries} and {\tt syntax} fields contain the
  libraries and syntax extensions defined by the package.

\end{itemize}

\subsubsection{Package configuration files: {\tt .config}}
\label{dotconfig}

\verb+$opam/OVERSION/config/NAME.config+ follows the syntax defined in
\S\ref{syntax}, with the following restrictions:

{\small
\begin{Verbatim}[frame=single]
<file>    :=
    opam-version: "1"
    <item>*
<item>    := <def> | <section>
<section> :=
    <kind> STRING {
      ?asmcomp:  [ STRING+ ]
      ?bytecomp: [ STRING+ ]
      ?asmlink : [ STRING+ ]
      ?bytelink: [ STRING+ ]
      ?requires: [ STRING+ ]
      <def>*
    }
<kind>    := library | syntax
<def>     := IDENT: BOOL
           | IDENT: STRING
           | IDENT: [ STRING+ ]
\end{Verbatim}
}

\verb+$NAME.config+ contains platform-dependent information which can
be useful for other libraries or syntax extensions that want to use
libraries defined in the package \verb+$NAME+.

\paragraph{Local and global variables}

The definitions ``{\tt IDENT: BOOL}'', ``{\tt IDENT: STRING}'' and ``{\tt IDENT:
  [ STRING+ ]}'', are used to defined variables associated to this
package, and are used to substitute variables in template files (see
\S\ref{subst}):

\begin{itemize}

\item \verb+%{$NAME:$VAR}%+ will refer to the variable \verb+$VAR+
  defined at the root of the configuration file \verb+$config/NAME.config+.

\item \verb+%{$NAME.$LIB:$VAR}%+ will refer to the variable \verb+$VAR+
  defined in the {\tt library} or {\tt syntax} section named
  \verb+$LIB+ in the configuration file \verb+$config/$NAME.config+.

\end{itemize}

\paragraph{Library and syntax sections}

Each {\tt library} and {\tt syntax} section defines an OCaml library
and the specific compilation flags to enable when using and linking
with this library.

The distinction between libraries and syntax extensions is only useful
at compile time to know whether the options should be used as
compilation or pre-processing arguments (ie. should they go on the
compiler command line or should they be passed to the {\tt -pp}
option). This is the responsibility of the build tool to do the right
thing and the {\tt <kind>} of sections is only used for documentation
purposes in OPAM. \\

The available options are:

\begin{itemize}
\item {\tt asmcomp} are compilation options to give to the native
  compiler (when using the {\tt -c} option)
\item {\tt bytecomp} are compilation options to give to the bytecode
  compiler (when using the {\tt -c} option)
\item {\tt asmlink} are linking options to give to the native compiler 
\item {\tt bytlink} are linking options to give to the bytecode
  compiler
\item {\tt requires} is the list of libraries and syntax extensions
  the current block is depending on. The full list of compilation
  and linking options is built by looking at the transitive closure of
  dependencies. The contents of {\tt dep}s is the list of libraries or
  syntax extension the current section depends on. Note that we do not
  refer here to any package name, as multiple packages can expose libraries
  with the same name and interface and thus we want the user to be able
  to switch between them easily.
\end{itemize}

\subsubsection{Package installation files: {\tt .install}}
\label{dotinstall}

\verb+$opam/OVERSION/install/NAME.install+ follows the syntax defined
in \S\ref{syntax} with the following restrictions:

{\small
\begin{Verbatim}[frame=single]
<file> :=
    opam-version: "1"
    ?lib:  [ STRING+ ]
    ?bin:  [ <mv>+ ]
    ?doc:  [ STRING+ ]
    ?misc: [ <mv>+ ]

<mv> := STRING
      | STRING { STRING }
\end{Verbatim}
}

\begin{itemize}
\item Files listed under {\tt lib} are copied to \verb+$lib/$NAME/+.
\item Files listed under {\tt bin} are copied to \verb+$bin/+ (they
  can be renamed using \verb+$SRC { $DST }+; in this case \verb+$SRC+
  should be a simple filename, ie. it should not start with a
  directory name).
\item Files listed under {\tt doc} are copied to \verb+$doc/$NAME/+.
\item Files listed under {\tt misc} should be processed as follows:
  for each pair \verb+$SRC { $DST }+, the tool should ask the user if
  he wants to install \verb+$SRC+ to the absolute path \verb+$DST+.
\end{itemize}

\subsubsection{Substitution files}
\label{subst}

All of the previous files can be generated using a special mode of
{\tt opam} which can perform tests and substitutes variables (see
\S\ref{opam-config} for the exact command to run). Substitution files
contains some templates which will be replaced by some contents. The
syntax of templates is the following:

\begin{itemize}

\item templates such as \verb+%{$NAME:$VAR}%+ are replaced by the value
  of the variable \verb+$VAR+ defined at the root of the file
  \verb+$config/NAME.config+.

\item templates such as \verb+%{$NAME.$LIB:$VAR}%+ are replaced by the
  value of the variable \verb+$VAR+ defined in the \verb+$LIB+ section
  in the file \verb+$config/PACKAGE.config+

%% \item templates such as \verb+%{IF $test %{ $then %} THEN %{ $else %}}+ are
%%    replaced by \verb+$then+ if \verb+$test+ is either:
%% \begin{itemize}
%% \item the ident {\tt true}
%% \item or a (global or local) variable whose value is the ident {\tt
%%   true}
%% \item or an expression \verb+$var1 = $var2+ or \verb+$var = STRING+ 
%%   where the contents of both sides of {\tt =} evaluates to the same
%%   value.
%% \end{itemize}
%% Otherwise, the template is replaced by \verb+$else+.

\end{itemize}

\subsection{Client commands}

\subsubsection{Creating a fresh client state}
\label{opam-init}

When an end-user starts OPAM for the first time, he needs to
initialize \verb+$opam/+ in a consistent state. In order to do so, he
should run:

\begin{verbatim}
    $ opam init [-kind $KIND] $REPO $ADDRESS
\end{verbatim}

Where:
\begin{itemize}
\item \verb+$KIND+ is the kind of OPAM repository (default is {\tt
  http});
\item \verb+$REPO+ is the name of the repository (default is {\tt
  default}); and
\item \verb+ADDRESS+ is the repository address (default is
  \verb+http://opam.ocamlpro.com/pub+).
\end{itemize}

This command will:

\begin{enumerate}

\item Create the file \verb+$opam/config+ (as specified in
  \S\ref{dotconfig})

\item Create an empty \verb+$opam/$OVERSION/installed+ file,
  \verb+$OVERSION+ %being the result of ``{\tt ocamlc -version}''.
  is the version from the OCaml used to compile \verb+$opam+.
  In particular, we will not fail now 
  if there is no \verb+ocamlc+ in \verb+$path+.

\item Initialize \verb+$opam/repo/$REPO+ by running
  \verb+'opam-$KIND-init $ADDRESS'+ (see \S\ref{script-update}). If
  the script cannot be found in the path, the command should be
  canceled and should return a well-defined error.

\item Symlink all OPAM and description files (ie. create a symbolic
  link from every file in \verb+$opam/repo/$REPO/opam/+ to
  \verb+$opam/opam/+ and from every file in
  \verb+$opam/repo/$REPO/descr/+ to \verb+$opam/descr/+).

\item Create \verb+$opam/repo/index+ and for each version
  \verb+$VERSION+ of package \verb+$NAME+ appearing in the repository,
  append the line \verb+'$REPO $NAME $VERSION'+ to the file.

\item Create the empty directories \verb+$opam/archives+,
  \verb+$lib/+, \verb+$bin/+ and \verb+$doc/+.

\end{enumerate}

\subsubsection{Listing packages}
\label{opam-list}

When an end-user wants to have information on all available packages,
he should run:

\begin{verbatim}
    $ opam list
\end{verbatim}

This command will parse \verb+$opam/$OVERSION/installed+ to know the
installed packages, and \verb+$opam/opam/*.opam+ to get all the
available packages. It will then build a summary of each packages. The
description of each package will be read in \verb+$opam/descr/+ if it
exists.

For instance, if {\tt batteries} version {\tt 1.1.3} is installed,
{\tt ounit} version {\tt 2.3+dev} is installed and {\tt camomille} is
not installed, then running the previous command should display:

\begin{verbatim}
    batteries   1.1.3  Batteries is a standard library replacement
    ounit     2.3+dev  Test framework
    camomille      --  Unicode support
\end{verbatim}

\subsubsection{Getting package info}

In case the end-user wants a more details view of a specific package,
he should run:

\begin{verbatim}
    $ opam info $NAME
\end{verbatim}

This command will parse \verb+$opam/$OVERSION/installed+ to get the
installed version of \verb+$NAME+, will process
\verb+$opam/repo/index+ to get the repository where the package comes
from and will look for \verb+$opam/opam/$NAME.*.opam+ to get available
versions of \verb+$NAME+. It can then display:

\begin{verbatim}
    package: $NAME
    version: $VERSION                 # '--' if not installed
    versions: $VERSION1, $VERSION2, ...
    libraries: $LIB1, $LIB2, ...
    syntax: $SYNTAX1, $SYNTAX2, ...
    repository: $REPO
    description:
      $SYNOPSIS

      $LINE1
      $LINE2
      $LINE3
      ...
\end{verbatim}

\subsubsection{Installing a package}
\label{opam-install}

When an end-user wants to install a new package, he should run:

\begin{verbatim}
    $ opam install $NAME
\end{verbatim}

This command will:

\begin{enumerate}

\item Compute the transitive closure of dependencies and conflicts of
  packages using the dependency solver (see \S\ref{deps}). If the
  dependency solver returns more than one answer, the tool will ask
  the user to pick one, otherwise it will proceed directly. The
  dependency solver should also mark the packages to recompile.

\item The dependency solver sorts the collections of packages in
  topological order. Then, for each of them do:

\begin{enumerate}

\item Check whether the package is already installed by looking for
  the line \verb+$NAME $VERSION+ in \verb+$opam/$OVERSION/installed+.
  If not, then:

\item Look into the archive cache to see whether it has already been
  downloaded. The cache location is:
  \verb+$opam/archives/$NAME.VERSION.tar.gz+

\item If not, process \verb+$opam/repo/index/+ to get the repository
  \verb+$REPO+ where the archive is available, get the repository kind
  by looking at \verb+$opam/repo/$REPO/kind+ and then ask the
  repository to download the archive by calling
  \verb+opam-$KIND-download+ (see \S\ref{script-download}).

  Once this is done, symlink the archive in \verb+$opam/archives+.

\item Decompress the archive into \verb+$build/$NAME.$VERSION/+.

\item Substitute the required files.

\item Run the list of commands to build the package with \verb+$bin+
  in the path.

\item Process \verb+$build/$NAME.$VERSION/$NAME.install+ to install
  the created files. The file format is described in \S\ref{dotinstall}.

\item Install the installation file
  \verb+$build/$NAME.$VERSION/$NAME.install+ in \verb+$install/+ and
  the configuration file \verb+$build/$NAME.$VERSION/$NAME.config+ in
  \verb+$config/+.

\end{enumerate}
\end{enumerate}

\subsubsection{Updating index files}
\label{opam-update}

When an end-user wants to know what are the latest packages available,
he will write:

\begin{verbatim}
    $ opam update
\end{verbatim}

This command will follow the following steps:

\begin{itemize}

\item For each repositories in \verb+$opam/config+, run the
  appropriate \verb+opam-$KIND-update+ script (see
  \S\ref{script-update}).

\item For each repositories in \verb+$opam/config+, process
  \verb+$opam/repo/$REPO/updated+ and update \verb+$opam/repo/index+,
  \verb+$opam/opam/+ and \verb+$opam/desc+ accordingly (ie. add the
  right lines in \verb+$opam/repo/index+ and create the missing
  symlinks). Here, the order in which the repositories are specified
  is important: the first repository containing a given version for a
  package will be the one providing it (this can be changed manually
  by editing \verb+$opam/repo/index+ later).

\item For each line \verb+$REPO $NAME $VERSION+ in
  \verb+$opam/repo/index+, if the version \verb+$VERSION+ of package
  \verb+$NAME+ has been modified upstream (ie. if the line
  \verb+$NAME $VERSION+ appears in \verb+$opam/repo/$REPO/$updated+)
  and if the package is already installed (ie. it appears in
  \verb+opam/$OVERSION/installed+), then update
  \verb+$opam/$OVERSION/reinstall+ accordingly (for each compiler
  version \verb+$OVERSION+).

  Packages in \verb+$opam/$OVERSION/reinstall+ will be reinstalled (or
  upgraded if a new version is available) on the next {\tt opam
    upgrade} (see \S\ref{opam-upgrade}), with \verb+$OVERSION+ being
  the current compiler version when the upgrade command is run.

\item Delete each \verb+$opam/repo/$REPO/$updated+

\end{itemize}

\subsubsection{Upgrading installed packages}
\label{opam-upgrade}

When an end-user wants to upgrade the packages installed on his host,
he will write:

\begin{verbatim}
    $ opam upgrade
\end{verbatim}

This command will:

\begin{itemize}

\item Call the dependency solver (see \S\ref{deps}) to
find a consistent state where {\bf most} of the installed packages are
upgraded to their latest version. Moreover, packages listed in
\verb+$opam/$OVERSION/reinstall+ will be reinstalled (or upgraded if a new
version is available). It will install each non-installed packages in
topological order, similar to what it is done during the install step,
See \S\ref{opam-install}.

\item Once this is done the command will delete
  \verb+$opam/$OVERSION/reinstall+.

\end{itemize}

\subsubsection{Getting package configuration}
\label{opam-config}

The first version of OPAM contains the minimal information to be able
to use installed libraries. In order to do so, the end-user (or the
packager) should run:

\begin{verbatim}
    $ opam config -list-vars
    $ opam config -var $NAME:$VAR
    $ opam config -var $NAME.$LIB:$VAR
    $ opam config -subst $FILENAME+
    $ opam config [-r] -I        $NAME+
    $ opam config [-r] -bytecomp $NAME.$LIB+
    $ opam config [-r] -asmcomp  $NAME.$LIB+
    $ opam config [-r] -bytelink $NAME.$LIB+
    $ opam config [-r] -asmlink  $NAME.$LIB+
\end{verbatim}

\begin{itemize}
\item \verb+-list-vars+ will return the list of all variables defined
  in installed packages (see \S\ref{dotconfig})
\item \verb+-var $var+ will return the value associated to the
  variable \verb+$var+
\item \verb+-subst $FILENAME+ replace any occurrence of
  \verb+%{$NAME:$VAR}%+ and \verb+%{$NAME.$LIB:$VAR}%+ as specified in
  \S\ref{subst} in \verb+$FILENAME.in+ to create \verb+$FILENAME+.
\item \verb+-I $NAME+ will return the list of paths to include when
  compiling a project using the package \verb+$NAME+ (\verb+-r+ gives
  a result taking into account the transitive closure of
  dependencies).
\item \verb+-bytecomp+, \verb+-asmcomp+, \verb+-bytelink+ and
  \verb+-asmlink+ return the associated value for the section
  \verb+$LIB+ in the file \verb+$config/$NAME.config+ (\verb+-r+ gives
  a result taking into account the transitive closure of all
  dependencies).
\end{itemize}


\subsubsection{Uploading packages}
\label{opam-upload}

When a packager wants to create a package, he should:

\begin{enumerate}

\item create \verb+$package/$NAME.$VERSION.opam+ containing in the format
  specified in \S\ref{dotopam}.

\item create a file describing the package

\item make sure the build scripts:
\begin{itemize}
\item build the libraries and packages advertised in
  \verb+$package/$NAME.$VERSION.opam+
\item generates a valid \verb+$package/$NAME.install+ containing the
  list of files to install (the file format is described in
  \ref{dotinstall}).
\item generates a valid \verb+$package/$NAME.config+ containing the
  configuration flags for libraries exported by this package (the file
  format is described in \ref{dotconfig}).
\end{itemize}

\item create an archive \verb+$NAME.$VERSION.tar.gz+ with the sources he
  wants to distribute.

\item run the following command:

\begin{verbatim}
    $ opam upload -opam $OPAM -descr $DESCR -archive $ARCHIVE -repo $REPO
\end{verbatim}

This command will parse \verb+$OPAM+ to get the package name and
version and it will:
\begin{itemize}
\item move \verb+$OPAM+ to \verb+$opam/repo/$REPO/upload/$NAME.$VERSION.opam+
\item move \verb+$DESCR+ to \verb+$opam/repo/$REPO/descr/$NAME.$VERSION+
\item move \verb+$ARCHIVE+ to \verb+$opam/repo/$REPO/archives/$NAME.$VERSION.tar.gz+
\end{itemize}

It will then call \verb+opam-$KIND-upload+ to upload the files
upstream (see \S\ref{script-upload}).

This command will work only for writable repositories; in case of
errors (for instance if the repository is read-only) the script
returns a non-zero exit code and stores the error message in
\verb+$opam/repo/$REPO/error+ (see \S\ref{script-error}).

\end{enumerate}

\subsubsection{Removing packages}
\label{opam-remove}

When the user wants to remove a package, he should write:

\begin{verbatim}
    $ opam remove $NAME
\end{verbatim}

This command will check whether the package \verb+$NAME+ is installed,
and if yes, it will display to the user the list packages that will be
uninstalled (ie. the transitive closure of all forward-dependencies).
If the user accepts the list, all the packages should be uninstalled,
and the client state should be let in a consistent state.

\subsubsection{Managing OPAM repository}

When the user wants to manage OPAM repositories, he should write:

\begin{verbatim}
    $ opam repository -list
    $ opam repository -rm $REPO
    $ opam repository -add [-kind $KIND] $REPO $ADRESS
\end{verbatim}

\begin{itemize}
\item \verb+-list+ lists the current repositories by looking at
  \verb+$opam/config+
\item \verb+-rm $REPO+ deletes \verb+$opam/repo/$REPO+ and removes
  \verb+$REPO+ from the {\tt repositories} list in \verb+$opam/config+.
  Then, for each package in \verb+$opam/repo/index+ it updates the link
  between packages and repositories (ie. it either deletes packages or
  symlink them to the new repository containing the package).

\item \verb+-add [-kind $KIND] $REPO $ADDRESS+ initializes
  \verb+$REPO+ as described in \S\ref{opam-init}.

\end{itemize}

\subsubsection{Dependency solver}
\label{deps}

Dependency solving is a hard problem and we do not plan to start from
scratch implementing a new SAT solver. Thus our plan to integrate (as
a library) the Debian depency solver for CUDF files, which is written
in OCaml.

\begin{itemize}
\item the dependency solver should run on the client; and
\item the dependency solver should take as input a list of packages
  (with some optional version information) the user wants to install,
  upgrade and remove and it should return a consistent list of
  packages (with version numbers) to install, upgrade, recompile and
  remove.
\end{itemize}

\subsection{OPAM repository scripts}
\label{scripts}

As stated above, OPAM repositories can be of different kinds. A given
kind \verb+$KIND+ has associated scripts \verb+opam-$KIND-init+,
\verb+opam-$KIND-update+, \verb+opam-$KIND-download+ and
\verb+opam-$KIND-upload+ which should behave in the way specified
below.

\subsubsection{Init script}
\label{script-init}

\begin{verbatim}
    $ opam-$KIND-init $ADDRESS
\end{verbatim}

This script should contact the OPAM repository located at address
\verb+$ADDRESS+ and populate its local state (ie. \verb+./opam/+,
\verb+./descr/+ and \verb+archives+) as specified in
\S\ref{state-repo}.

\subsubsection{Update script}
\label{script-update}

\begin{verbatim}
    $ opam-$KIND-update $ADDRESS
\end{verbatim}

This script should get the list of newly available packages, by
contacting the OPAM repository located at address \verb+$ADDRESS+
and update \verb+$./opam+ and \verb+./descr+.

It should then create (or update) \verb+./updated+ to reflect the
newly available packages and the one who changed uptstream..

\subsubsection{Download script}
\label{script-download}

\begin{verbatim}
    $ opam-$KIND-download $ADDRESS $NAME.$VERSION
\end{verbatim}

This script should get \verb+$NAME.$VERSION.tar.gz+ from the OPAM
repository located at address \verb+ADDRESS+ and store it into
\verb+$opam/repo/$REPO/archives/+ (this should overwrite any
previously existing file if one is already there).

\subsubsection{Upload script}
\label{script-upload}

\begin{verbatim}
    $ opam-$KIND-init $opam/repo/$REPO
\end{verbatim}

This script should send the contents of
\verb+$opam/repo/$REPO/upload/+ to the repository whose address is
stored in \verb+$opam/repo/$REPO/address+ and remove the folder
contents when it's done.

\subsubsection{Error handling}
\label{script-error}

In case any of the above scripts return with a non-zero exit code, the
full error message will be stored into \verb+opam/repo/$REPO/error+

\subsubsection{RSYNC repository}

The first Milestone includes the necessary scripts to support {\tt
  rsync} repositories. In order to do so:

\begin{itemize}

\item The OPAM repository is a filesystem similar to what
  is described in \verb+$opam/repo/$REPO+.

\item \verb+opam-rsync-init $PATH $ADDRESS+ simply runs something like:

\begin{verbatim}
    $ cd $PATH && rsync [..] $ADDRESS/opam && rsync [..] $ADDRESS/descr
\end{verbatim}

\item \verb+opam-rsync-download $PATH $NAME $VERSION+ will simply run
  something like:

\begin{verbatim}
    $ cd $PATH && rsync [..] $ADDRESS/archives/$NAME.$VERSION.tar.gz
\end{verbatim}

\item \verb+opam-rsync-update $PATH+ will also run {\tt rsync} and
  filter the command output to populate
  \verb+$opam/repo/$REPO/updated+ accordingly.

\item \verb+opam-rsync-upload $PATH+ will return an error if the OPAM
  repository is not writable (for instance if the repository if an
  HTTP server). In this case the server state can be updated manually
  by copying the new files in the right place on the server
  filesystem.

\end{itemize}

\section{Milestone 2: OPAM server}
\label{server}

The only OPAM repository kind available in Milestone 1 is {\tt rsync}.

This second milestone describes a new kind of repository: {\tt
  server}. The OPAM server uses a custom binary protocol between a
client (the {\tt opam-server-*} scripts) and a server ({\tt
  opam-server} which is run on the server). This is intended to be the
preferred way to upload new packages to OPAM repositories as it does
not requires any external authentification mechanisms.

\subsection{Server state}
\label{server-state}

In this section, {\tt \$opamserver} refers to the filesystem subtree
containing the server state. Default directory is {\tt
  \$home/.opam-server}. The state is similar to what is described in
\S\ref{state-global}:

\begin{itemize}
\item \verb+$opamserver/opam/$NAME.$VERSION.opam+ is the OPAM
  specification for the package \verb+$NAME+ with version
  \verb+$VERSION+ (which might not be installed). The format of OPAM
  files is described in \S\ref{dotopam}.

\item \verb+$opamserver/descr/$NAME.$VERSION+ contains the description
  for the version \verb+$VERSION+ of package \verb+$NAME+ (which might
  not be installed). The first line of this file is the package
  synopsis.

\item \verb+$opamserver/archives/$NAME.$VERSION.tar.gz+ contains the
  source archives for the version \verb+$VERSION+ of package
  \verb+$NAME+.

\end{itemize}

\subsection{Binary Protocol}
\label{binary-protocol}

The protocol is very simple, there is one kind of messages
corresponding to each basic client actions (which will be used by the
{\tt opam-server-*} scripts). The only ``complex'' parts resides in
the basic authentication mechanisms: the server stores a key for each
package (every version shares the same key); the key is sent to the
client the first time a new package is uploaded and key consistency is
checked each time a new version of the package is uploaded.

\begin{itemize}

\item Communication between clients and servers always start by an
hand-shake to agree on the protocol version.

\item All the basic values (names, versions and binary data) are
  represented as in OCaml strings and are marshaled as an 64-bits
  integer (the string size) followed by the string contents.

\item Messages are marshaled using a simple binary protocol: the first
  byte represents the message number, and then each string argument is
  stacked sequentially.
  The list of messages {\em from the client to server} is:

{\small
\begin{tabular}{|l|l|l|l|}
\hline
\# & Client-to-Server Message & Contents & Description \\
\hline
\hline
0 & \verb+ClientVersion+ & \verb+version: string+ & Send the client version to the server \\
\hline
1 & \verb+GetList+ & -- & Ask for the list of all available OPAM files \\
\hline
2 & \verb+GetOPAM+ & \verb+name   : string+ & Ask for the binary representation of \\
  &                & \verb+version: string+ & a given OPAM file \\
\hline
3 & \verb+GetDescr+ & \verb+name   : string+ & Ask for the binary representation of \\
  &                 & \verb+version: string+ & a given description file \\
\hline
4 & \verb+GetArchive+ & \verb+name   : string+ & Ask for the binary representation of \\
  &                   & \verb+version: string+ & a given archive file \\
\hline
5 & \verb+NewPackage+ & \verb+name   : string+ & Create a new package on the server. \\
  &                   & \verb+version: string+ & The client should provide the OPAM and \\
  &                   & \verb+opam   : string+ & descr files and the source archive. \\
  &                   & \verb+descr  : string+ & \\
  &                   & \verb+archive: string+ & \\
\hline
6 & \verb+NewVersion+ & \verb+name   : string+ & Upload a new version of an already existing \\
  &                   & \verb+version: string+ & package on the server. The client \\
  &                   & \verb+opam   : string+ & should also provide a security key\\
  &                   & \verb+descr  : string+ & \\
  &                   & \verb+archive: string+ & \\
  &                   & \verb+key    : string+ & \\
\hline
\end{tabular}
}

\item Answers from the server are encoded in the same way (ie, a byte
  for the message number, followed by the sequential marshaling of the
  arguments). Pairs are stacked sequentially. List are encoded by
  stacking first the list length in a 64-bit integer and then all
  the elements of the list in sequential order. The list of messages
  {\em from servers to clients} is:

{\small
\begin{tabular}{|l|l|l|l|}
\hline
\# & Server-to-Client Message & Contents & Description \\
\hline
\hline
0 & \verb+ServerVersion+ & \verb+version: string+  & Return the server version \\
\hline
1 & \verb+PackageList+  & \verb+list   : (string*string) list+ & Return the list of available \\
  &                     &                                      & package names and versions \\
\hline
2 & \verb+OPAM+         & \verb+opam   : string+ & Return an OPAM file \\
\hline
3 & \verb+Descr+        & \verb+descr  : string+ & Return an OPAM file \\
\hline
4 & \verb+Archive+      & \verb+archive: string+ & Return an archive file \\
\hline
5 & \verb+Key+          & \verb+key    : string+ & Return a security key \\
\hline
6 & \verb+OK+           & --                     & The update went OK \\
\hline
7 & \verb+Error+        & \verb+error  : string+ & An error occurred \\
\hline
\end{tabular}
}

\end{itemize}

Note that when an error is raised by an arbitrary function
 at server side, the client receives \verb|Error|.

\subsection{Server authentication}

The server should be able to ask for basic credential proofs. The
protocol can be sketched as follows:

\begin{itemize}

\item packagers store keys in {\tt \$opam/keys/NAME}. These keys are
  random strings of size 128.

\item the server stores key hashes in {\tt
  \$opamserver/hashes/NAME}.

\item when a packager uploads a fresh package the server returns a
  random key. Clients then stores that key in {\tt \$opam/keys/NAME}.

\item when a packager wants to uplaod a new version of an existing
  package, the client also sends the key. The server then checks
  whether the hash of the key is equal to the one stored in {\tt
    \$opamserver/hashes/NAME}; if yes, it updates the
  package and return {\tt OK}, if no if it returns an {\tt Error}.

\end{itemize}

\subsection{Scripts}

The initialization script: {\tt opam-server-init}, the update script:
{\tt opam-server-update}, the download script: {\tt
  opam-server-download} and the upload script:{\tt opam-server-upload} will work as
expected, using the binary protocol defined in
\S\ref{binary-protocol} in order to correctly update the client state as
defined in \S\ref{scripts}.


\section{Milestone 3: Multiple compiler versions}

This milestone focus on the support of multiple compiler versions.

\subsection{Compiler Description Files}

For each compiler version {\tt OVERSION}, the client state will be
extended with the following files:

\begin{itemize}
\item {\tt \$opam/compilers/OVERSION.comp}
\end{itemize}

The syntax of {\tt .comp} files follows the one described in
\S\ref{syntax} with the following restrictions:

{
\begin{Verbatim}[frame=single]
<file> :=
    opam-version: "1"
    src:        STRING
    make:       [ STRING+ ]
    ?patches:   [ STRING+ ]
    ?configure: [ STRING+ ]
    ?bytecomp:  [ STRING+ ]
    ?asmcomp:   [ STRING+ ]
    ?bytelink:  [ STRING+ ]
    ?asmlink:   [ STRING+ ]
    ?packages:  [ STRING+ ]
    ?requires:  [ STRING+ ]
    ?pp:        [ <ppflag>+ ]

<ppflag> := CAMLP4 { <dep>+ }
          | <flag>
\end{Verbatim}
}
\begin{itemize}

\item {\tt src} is the location where this version can be downloaded. It can be:
\begin{itemize}
\item an archive available in the local filesystem
\item an archive available via {\tt http} or {\tt ftp}
\item a version-controlled repository under {\tt svn} or {\tt git}
  (with the expectation that these tools are installed on the user host).
\end{itemize}

\item {\tt patches} are optional patch addresses, available via {\tt http}, {\tt ftp}
  or locally on the filesystem.

\item {\tt configure} are the optional flags to pass to the configure
  script. The order is relevant: {\tt --prefix=\$opam/OVERSION/} will be automatically 
  added at the end to these options.
  Remark that if these flags contain {\tt --bindir}, {\tt --libdir}, 
  and {\tt --mandir}, then every {\tt --prefix} will be ignored by {\tt configure}.

\item {\tt make} are the flags to pass to {\tt make}. 
  It must at least contain some target like {\tt world} or {\tt world.opt}.

\item {\tt packages} is the list of packages

\item {\tt bytecomp}, {\tt asmcomp}, {\tt bytelink} and {\tt asmlink}
  are the compilation and linking flags to pass to the OCaml
  compiler. They will be taken into account by the \verb+opam config+
  command (see \S\ref{opam-config}).

\item {\tt packages} is the list of packages to install just after the
  compiler installation finished. These libraries will not consider
  what is in the {\tt requires} nor {\tt pp} (as {\tt requires} and
  {\tt pp} might want to use things already installed with {\tt
    packages}).

\item {\tt requires} is a list of libraries and syntax extensions
  dependencies which will be added to every packages installed with
  this compiler. The libraries and syntax extensions should be present
  in packages defined in {\tt packages}, othewise an error should be
  thrown.

\item {\tt pp} is the command to use with the {\tt -pp} command-line
  argument. It is either a single flag or a {\tt camlp4} command,
  such as {\tt CAMLP4 [ pp-trace ]}: this will look for the
  compilation flags for the syntax extension {\tt pp-trace} and expand
  the {\tt camlp4} command-line accordingly. All the packages should
  be present in {\tt packages}.

\end{itemize}

For instance the file, {\tt 3.12.1+memprof.comp} describes OCaml,
version $3.12.1$ with the memory profiling patch enabled:

\begin{verbatim}
src:       "http://caml.inria.fr/pub/distrib/ocaml-3.12/ocaml-3.12.1.tar.gz"
make:     [ "world" "world.opt" ]
patches:   [ "http://bozman.cagdas.free.fr/documents/ocamlmemprof-3.12.0.patch" ]
\end{verbatim}

And the file {\tt trunk-g-notk-byte.comp} describes OCaml from SVN
trunk, with no {\em tk} support and only in bytecode, and all the
libraries built with {\tt -g}:

\begin{verbatim}
src:       "http://caml.inria.fr/pub/distrib/ocaml-3.12/ocaml-3.12.1.tar.gz"
configure: [ "-no-tk" ]
make:     [ "world" ]
bytecomp:  [ "-g" ]
bytelink:  [ "-g" ]
\end{verbatim}

\subsection{Switching compiler version}

If the user wants to switch to an other compiler version, he should run:

\begin{verbatim}
    $ opam switch $OVERSION
\end{verbatim}

This command will:

\begin{itemize}
\item Look for an existing \verb+$opam/$OVERSION+ directory. If it
  exists, then change the {\tt ocaml-version} content to
  \verb+$OVERSION+ in  \verb+$opam/config+.

\item If it does not exist, look for an existing
  \verb+$opam/compilers/OVERSION.comp+. If the file does not exists,
  the command will fail with a well-defined error.

\item If the file exist, then build the new compiler with the right
  options (and pass \verb+--prefix $opam/$OVERSION+ to
  \verb+./configure+) and initialize everything in \verb+$opam/+ 
  in a consistent state as if ``\verb+opam init+'' has just been called.

\end{itemize}

\section{Milestone 4: OPAM and git}

\subsection{OPAM repository under git}

\subsection{Package repository under git}


\section{Milestone 9: Parallel Build}

\section{Milestone 8: Version Pinning}


\end{document}
